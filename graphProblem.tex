\documentclass{article}

% Packages
\usepackage[utf8]{inputenc} % Input encoding
\usepackage[T1]{fontenc} % Font encoding
\usepackage{lipsum} % Dummy text package

\usepackage{listings} % For code listings
\usepackage{xcolor} % For custom colors
\usepackage [colorlinks=true, linkcolor=blue]{hyperref}
\usepackage[margin=1 cm]{geometry} % Adjust the margin values as needed
\usepackage{float}

% Define code listing style
\lstdefinestyle{cppStyle}{
  language=C++,
  basicstyle=\small\ttfamily,
  keywordstyle=\color{blue},
  commentstyle=\color{green!60!black},
  stringstyle=\color{red},
  showstringspaces=false,
  breaklines=true,
  frame=single,
  numbers=left,
  numberstyle=\tiny,
  numbersep=5pt,
  tabsize=2
}

% Title and author information
\title{Problem Set }
\author{sharif}
\date{\today} % Use \date{} for no date or specify a specific date

% Document content
\begin{document}

\maketitle


\lstset{style=cppStyle}

\section{Wormholes} \href{https://onlinejudge.org/index.php?option=onlinejudge&Itemid=8&category=551&page=show_problem&problem=499}{Link} \\

This problem is based on finding a \textbf{negative cycle} in a graph.\\
ALgorithm used: \textbf{Bellman ford }\\
\\
\\
\\

{\small
\begin{lstlisting}[caption={},aboveskip=0pt, belowskip=0pt]

#include<bits/stdc++.h>
using namespace std ;


 bool find_negative_cycle(vector<pair<int,int>>adj[],int noVertices ,int source)
{
    vector<int>distance(noVertices,numeric_limits<int>::max());
    distance[source]=0 ;

    //Relaxation
    for(int i=0 ;i<noVertices-1;i++)
    {
        for(int u=0;u<noVertices;u++)
        {
            for(auto edges: adj[u])
            {
                int v=edges.first ;
                int weight=edges.second ;
                if(distance[u]!=numeric_limits<int>::max() && distance[u]+weight<distance[v])
                {
                    distance[v]=distance[u]+weight ;
                }
            }
        }
    }

    //additional iteration to check negative cycle
    for(int i=0 ;i<noVertices ;i++)
    {
        for(auto edges: adj[i])
            {
                int v=edges.first ;
                int weight=edges.second ;
                if(distance[i]!=numeric_limits<int>::max() && distance[i]+weight<distance[v])
                {
                    return true ;
                }
            }
    }
    return false ;

}


int main()
{

int t ,n,m;//star system , wormholes
scanf("%d",&t);
while(t--)
{
 scanf("%d%d",&n,&m);
 vector<pair<int,int>>adj[n+10] ;
 for(int i=0 ;i<m ;i++)
 {
     int u,v,w ;
     scanf("%d%d%d",&u,&v,&w);
     adj[u].push_back({v,w});

 }

 if(find_negative_cycle(adj,n,0))
 {
     printf("possible\n");
 }
 else
 {
     printf("not possible\n");
 }

}

return 0 ;
}



\end{lstlisting}

}



\section{King Escape} \href{https://codeforces.com/problemset/problem/1033/A}{Link} \\


ALgorithm used: \textbf{dfs }


\begin{figure}[H] 
{
\small
\begin{lstlisting}[caption={},aboveskip=0pt, belowskip=0pt]

#include<bits/stdc++.h>
using namespace std;
const int N=1e3+10 ;

int visited[N][N] ;
int n,ax,ay,bx,by,cx,cy ;

bool checkValidMove(int nx,int ny) //new x and new y
{
    if(nx==ax || ny==ay) // check horizontal and vertical
    {
        return true ;
    }
    if(abs(nx-ax)==abs(ny-ay)) //check queen and king are same diagonal or not
    {
        return true ;
    }

    return false ;
}

void dfs(int x,int y) {

    int dx[8] = {1, 1, 1, 0, 0, -1, -1, -1};
    int dy[8] = {1, 0, -1, 1, -1, 1, 0, -1};

        if(x<1 || x>n || y<1 || y>n)
        {
            return ;
        }
        if(visited[x][y])
        {
            return  ;
        }

        if(checkValidMove(x,y))
        {
            return ;
        }
    visited[x][y]=1 ;

        for (int i = 0; i < 8; ++i) {
            int nx = x + dx[i];
            int ny = y + dy[i];

            dfs(nx,ny);//check for new cell

        }
}

int main() {
cin >> n;
   // queen location
    cin >> ax >> ay;
    // king location
    cin >> bx >> by;
    // target location
    cin >> cx >> cy;

    dfs(bx,by);
    if (visited[cx][cy])
        cout << "YES" << endl;
    else
        cout << "NO" << endl;

    return 0;
}


\end{lstlisting}

}
\end{figure}


\section{ leetCode-Find if Path Exists in Graph
} \href{https://leetcode.com/problems/find-if-path-exists-in-graph/description/}{Link} \\

This problem is based on finding a \textbf{connected component } in a graph.\\
ALgorithm used: \textbf{dfs }  \\ \\

{\small
\begin{lstlisting}[caption={},aboveskip=0pt, belowskip=0pt]

class Solution {

     void dfs(int source ,vector<int>adj[], vector<bool>&visit )
    {
        visit[source]=true ;
        for(auto child: adj[source]) //traverse adjacent nodes  of source 
        {
            if(!visit[child])
            {
                visit[child]=true ;
                dfs(child,adj,visit);
            }
        }
    }
public:
    bool validPath(int n, vector<vector<int>>& edges, int source, int destination) {
        
        int x=edges.size() ;
        vector<int>adj[n];
        vector<bool>visit(n,false) ;

        for(int i=0 ;i<x ;i++)
        {
            adj[edges[i][0]].push_back(edges[i][1]);
            adj[edges[i][1]].push_back(edges[i][0]);

        }

        
                dfs(source,adj,visit); // traverse from source if found destination return true 

                if( (visit[source] && !visit[destination]) ||  (!visit[source] &&visit[destination]))
                {
                    return false ;
                }
            
        
         return true ;


    }
};

\end{lstlisting}
}



\section{CodeForces-Bmail Computer Network} \href{https://codeforces.com/problemset/problem/1057/A}{Link} \\

This problem is based on finding a \textbf{Connected path from source to destination } in a graph.\\
Algorithm used: \textbf{BFS}\\
\\
\\
\\
\\
\\
\\



{\small
\begin{lstlisting}[style=cppStyle]
#include<bits/stdc++.h>
using namespace std ;
const int N=200000 ;
int visited[N];

void bfs(vector<int>graph[],int parent[],int source)
{
    queue<int>q ;
    q.push(source);
    visited[source]=1 ;
    parent[source]=-1 ;
    while(!q.empty())
    {
        int v=q.front();
        for(auto u: graph[v])
        {
            if(!visited[u])
            {
                parent[u]=v ;
                visited[u]=true ;
                q.push(u);

            }
        }
           q.pop() ;
    }

}

void path(int parent[],int src) // using recursion for finding parent ( destination to source)
{

    if(parent[src]==-1){
            printf("%d ",1);
        return ;
    }


   path(parent ,parent[src]);
   printf("%d ",src);
}

int main()
{
 int n ;
 cin>>n ;
 vector<int>graph[n+5];

 int parent[n]={} ;
 //8
 //1 1 2 2 3 2 5

 for(int i=2 ;i<=n ;i++)
 {
     int a;
     cin>>a ;
     graph[i].push_back(a);
     graph[a].push_back(i);
 }

 bfs(graph,parent,1) ;
 path(parent,n);


 //** you can use this path(parent,n) method  or this block of code to print path **

 /*
 int src=n ;
 set<int>path ;
 while(src!=-1)
 {
     path.insert(src);
     src=parent[src] ;
 }
 for(auto u: path)
    cout<<u<<" " ;*/


}

\end{lstlisting}
}


\section{LeetCode - Convert Sorted Array to Binary Search Tree } \href{https://leetcode.com/problems/convert-sorted-array-to-binary-search-tree/description/}{Link} \\

This problem is based on  \textbf{convert a array to binary search tree }\\
Algorithm used: \textbf{Binary search Tree}

\begin{figure}[H]
\begin{lstlisting}[style=cppStyle]
class Solution {
public:
    TreeNode* sortedArrayToBST(vector<int>& nums) {
        int size = nums.size();
        int half = size/2;
        
        if(size == 0)
            return NULL; // if no child

        
        TreeNode* root = new TreeNode(nums[half]);
        //create a root node 
        if(size == 1)
            return root;

          //left child ( start to half of nums vector )  
        vector<int>left(nums.begin(), nums.begin() + half);
        //right child( half to end  of nums vector )
        vector<int>right(nums.begin() + half + 1, nums.end());
        root->left = sortedArrayToBST(left);
        root->right = sortedArrayToBST(right);
        
        return root;
    }
};
\end{lstlisting}
\end{figure}



\section{Uva - Dividing coins } \href{https://onlinejudge.org/index.php?option=com_onlinejudge&Itemid=8&page=show_problem&problem=503}{Link} \\

This problem is based on  \textbf{ optimization technique }\\
Algorithm used: \textbf{0/1 knapSack}\\
\\
\\
\\
\\
\\


\begin{lstlisting}[style=cppStyle]

#include<bits/stdc++.h>
using namespace std ;
const int N = 110;
const int MAXN = 102*500;
int weight[N];
int dp[N][MAXN] ;

int sum ;

//used 2D array

int knapSack(int m)
{
    int max_weight=sum/2 ; // here max capacity will be half of  total weight

    for(int i=0; i <=max_weight; i++)
     dp[0][i] = 0;


    for(int i=1 ;i<=m;i++)
    {
        for(int w=0 ;w<=max_weight ;w++)
        {
            if(weight[i]<=w)
            {
               dp[i][w]=max(dp[i-1][w],(dp[i-1][w-weight[i]]+weight[i]));
            }
            else
            {
                dp[i][w]=dp[i-1][w] ;
            }

        }
    }


    return dp[m][max_weight];
}

//using 1D Array (memory optimization)
//int  knapSack(int m)
//{
//    int w=sum/2 ;
//  for(int i=1 ;i<=m ;i++)
//  {
//      for(int j=w ;j>0 ;j--)
//      {
//          if(weight[i]<=j)
//          {
//              dp[j]=max(dp[j],weight[i]+dp[j-weight[i]]);
//          }
//      }
//  }
//    return dp[w];
//}
//

int main()
{

int t ;
scanf("%d",&t);
while(t--)
{
int m ;
scanf("%d",&m);

sum=0 ;
for(int i=1 ;i<=m ;i++)
{
    scanf("%d",&weight[i]) ;
    sum+=weight[i] ;
}

  printf("%d\n", sum - 2*knapSack(m) );
}
return 0 ;
}
\end{lstlisting}






\section{Codeforces-KnapSack } \href{https://codeforces.com/problemset/problem/1446/A}{Link} \\

This problem is based on  \textbf{ greedy technique }\\
Algorithm used: \textbf{}\\

\begin{lstlisting}[style=cppStyle]
#include<bits/stdc++.h>
using namespace std ;


long long  total_weight ;
long long  oneItem ;
long long  weight ;
vector<int>items ;

int main()
{

    int t ;
    cin>>t ;
    while(t--)
    {
        long long n,w ;
        cin>>n>>w ;


        total_weight=0 ;
        oneItem=0 ;

        for(int i=1 ; i<=n ; i++)
        {
            cin>>weight ;
            if(weight <= w )
            {
                if(weight>=(w+1)/2)
                {
                    oneItem=i ;
                }
                else if ( total_weight < (w+1)/2)
                {
                    items.push_back(i);
                    total_weight+=weight ;
                }
            }

        }

        if(oneItem>0)
        {
            cout<<1<<'\n'<<oneItem ;

        }
        else
        {
            if(total_weight>=(w+1)/2)
            {
                cout<<items.size()<<endl ;
                for(auto u: items)
                    cout<<u<<" " ;

            }
            else
            {
                cout<<-1 ;
            }
        }

        cout<<endl;
        items.clear();

    }

}

\begin{figure}[H]
\end{figure}
\end{lstlisting}

\section{LeetCode-Is Subsequence } \href{https://leetcode.com/problems/is-subsequence/description/}{Link} \\

This problem is based on  \textbf{ Longest common subsequence  }\\
Algorithm used: \textbf{LCS}\\

\begin{lstlisting}[style=cppStyle]
class Solution {
public:
    bool isSubsequence(string s, string t) {

        int sub=s.size();
        int  original=t.size();
        
        int dp[sub+1][original+1];

        for(int i=0 ;i<=sub ;i++)
        dp[i][0]=0 ; //fill first row and first column with zero
        for(int i=0 ; i<=original;i++)
        dp[0][i]=0 ;

        for(int i=1 ;i<=sub;i++)
        {
            for(int j=1 ;j<=original ;j++)
            {
                if(s[i-1]==t[j-1])
                {
                    dp[i][j]=1+dp[i-1][j-1];//if match subsequence with original string 
                }
                else
                {
                    dp[i][j]=max(dp[i][j-1],dp[i-1][j]);// if not match 
                }
            }
        }

        if(dp[sub][original]==sub)
        return true ;
        else
        return false ;
        
    }
};
\end{lstlisting}


%\section{Methodology}
%\lipsum[2-3] % Dummy text



% Bibliography (if needed)
% \bibliographystyle{plain}
% \bibliography{references}

\end{document}
